\SetKwComment{tcf}{!}{}
\section{Introduction}
\par{For the density matrix which is associated with a determinant wave function of Slater type 
we can define the operator}
\begin{equation}
{\hat\rho}=\sum_{n=0}^{\infty}f_n\ket{n}\bra{n}
\end{equation}
where the occupancy for each state is given by $f_n$.
\par{In the statistical mechanics language this is called the first order
density matrix. The density matrix contains no more or less information
then the wavefunction (the determinant built from occupied states).
However if the occupancy numbers are allowed to be fractional, it carries information
about systems which are open, and can not be fully described by their own wavefunction.}
\par{For a system in its ground state we have to sum over the $N/2$ states of lowest energy
(we have assumed even number of electrons). In this case }
\begin{equation}
f_n=2, \quad n<N/2
\end{equation}
\par{The density operator can be generalised for finite temperature in a very straight forward way. Within the 
model of independent particles $f_n$ follows a \emph {Fermi distribution}}
\begin{equation}
f_{n}=2f_F(\epsilon_n)
\end{equation}
for a spinless model. For a spin model we have
\begin{equation}
f_{n}=f_F(\epsilon_n)
\end{equation}
\par{In both cases the $f_F(\epsilon)$ represents the well known Fermi distribution function}
\begin{equation}
\label{fermidistribution}
f_F(\epsilon)=\frac{1}{1+e^{\frac{\epsilon-\mu}{k_BT}}}
\end{equation}
where $k_B$ is Boltzmann constant, $T$ is the electronic temperature, $\epsilon$ the energy
and $\mu$ is the chemical potential. $\mu$ is defined in such a way to ensure that 
the total number of electrons $N$ is achieved
\begin{equation}
N=\sum_{n=0}^{\infty}2f_F(\epsilon_n)
\end{equation}
and without $2$ in the case of spin case.
\par{In a basis of local orbitals with}
\begin{equation}
\ket{n}=\sum_{I\mu}C_{I\mu}^{n}\ket{I\mu}
\end{equation}
we get
\begin{equation}
\hat{\rho}=\sum_{I\mu J\nu}\ket{I\mu}\rho^{I\mu J\nu}\bra{J \nu}
\end{equation}
where
\begin{equation}
\rho^{I\mu J\nu}=\sum_{n=0}^{\infty}f_nC_{I\mu}^{n}C_{J\mu}^{n*}
\end{equation}
The charge density is
\begin{equation}
\rho(\bm{r})=\sum_{I\mu J\nu}\bk{\bm{r}}{I\mu}\rho^{I\mu J\nu}\bk{J \nu}{\bm{r}}
\end{equation}
\par{Another quantity which will be needed is the entropy. The entropy is given by}
\begin{equation}
\bm{S}=-k_B\sum_{i=0}^{\infty}\left(f_F(\epsilon_i)\ln(f_F(\epsilon_i)+(1-f_F(\epsilon_i))\ln(1-f_F(\epsilon_i)\right)
\end{equation}

\section{Computer implementation}
\subsection{Spinless case}
\label{spinless}
\par{{\bf The problem.} Having a Hamiltonian $n \times n$. Let us consider its eigenvalues
$\epsilon_i, \; i=1,n$ such that $\epsilon_i < \epsilon_j$ if $i<j$ and its eigenvectors 
in a matrix $n \times n$ $a$ such that column $i$ contains the eigenvector associated 
with eigenvalue $\epsilon_i$. Find the density matrix $\rho$. We also know $q_{total}$, total charge.
}
\par{{\bf Solution:}}
\par{{\bf Step 1} finding the chemical potential, or Fermi level, $\mu$}

\begin{algorithm}[H]
\label{fermilevel}
\KwData{$\{\epsilon_i\}_{i=1,n},\; a(1:n,1:n),\;T,\;k_B,\; q_{total},\;maxit$}
\KwResult{finds chemical potential $\mu$}
$a=\epsilon_1-k_BT\ln(-1+\frac{100}{q_{total}})$\;
$b=\epsilon_n+k_BT\ln(-1+\frac{100}{q_{total}})$\;
$q=-1$\;
$k=0$\;
\While{$(2q \neq q_{total}) || (k < maxit)$}{
\tcf{$2q$ because of spin degeneracy}
$\mu=\frac{b-a}{2}+a$\;
$q=\sum_{i=1}^{n}f_F(\epsilon_i;\mu)$\;
\eIf{$2q-q_{total}>0$}{
$b=\mu$\;
}{
$a=\mu$\;
}
$k=k+1$\;
}
\If{k=maxit}{
call error("could not find $\mu$ to desired precision in $maxit$ iterations")\;
}
\caption{How to find chemical potential $\mu$}
\end{algorithm}

\par{The execution without error of the algorithm \gref{fermilevel} would give the 
$\mu$ consistent with the number of electrons. function $f_F$ returns the occupation number according
to Fermi distribution see eq. \gref{fermidistribution}.}

\par{{\bf Step2:} computing the density matrix elements.}
\par{To save computation time we employ some tricks (see algorithm \gref{matrixelements}).}
\begin{algorithm}
\label{matrixelements} 
\caption{Density matrix elements}
\KwData{$\{\epsilon_i\}_{i=1,n},\; a(1:n,1:n),\;T,\;k_B,\; \mu $}
\KwResult{density matrix $\rho$}
$uos=n$\;
$unos=0$\;
\For{i=1,n}{
$f(i)=f_F(\epsilon_i;\mu)$\;
\lIf(computes upper occupied state){$f(i)>q_{tolerance}$}{$uos=i$}
\tcf{computes upper occupied state with occupancy one}
\tcf{because we do not to compute $\sqrt{1}$}
\lIf{$|f(i)-1|<q_{tolerance}$}{$unos=i$}
}
\lFor{i=unos+1,uos}{$a(:,i)=\sqrt{f(i)}a(:,i)$}\;
$\rho=0$\;
compute $\rho(UpperTriangle)=a(:,1:uos)a(:,1:uos)^{\dagger*}$\;
\tcf{see algorithm \gref{uppertriangle}}
\tcf{populate lower triangle of $rho$}
\lFor{i=1,n}{$\rho(i+1:n,i)=(\rho(i,i+1:n))^*$}\;
\tcf{Compute the entropy $\bm{S}$}
$\bm{S}=0$\;
\For{i=unos+1,uos}{
$\bm{S}=\bm{S}-2k_B\left(f_F(\epsilon_i)\ln(f_F(\epsilon_i)+(1-f_F(\epsilon_i))\ln(1-f_F(\epsilon_i)\right)$\;
}
\end{algorithm}

\begin{algorithm}
\label{uppertriangle} 
\caption{Computes upper triangle part of density matrix $\rho$}
\KwData{$a(1:n,1:k)$}
\KwResult{upper triangle of density matrix $\rho$}
\tcf{Probably the best thing is to use ?herk from Blas Level 3}
\For{j=1,n}{
\For{l=1,k}{
\If{$a(j,l) \neq 0$}{
$tmp=(a(j,l))^*$\;
\For{i=1,j}{
$c(i,j)=c(i,j)+tmp*a(i,l)$\;
}
}
}
}
\end{algorithm}
\subsection{Colinear Spins Case}
\par{In the spin case the Hamiltonian is of the form}
\begin{equation}
\label{spinH}
H=\left(
\begin{array}{c|c}
\uparrow \uparrow & \uparrow \downarrow \\
\hline
\downarrow \uparrow & \downarrow \downarrow
\end{array}
\right)
\end{equation}
\par{$\uparrow \downarrow$ and $\downarrow \uparrow$ parts of the Hamiltonian are
zero for colinear spins. So we can split our Hamiltonian in two $n \times n$
Hamiltonians one for spin up and one for spin down. For each of them we can apply 
the method from section \gref{spinless}}.
\par{For spin up Hamiltonian $H^{\uparrow}$ we have $q_{total}^{\uparrow}$ and we get
$\mu^{\uparrow},\;\rho^{\uparrow},\;\bm{S}^{\uparrow}$. Similarly for spin down Hamiltonian $H^{\downarrow}$ 
we have $q_{total}^{\downarrow}$ and we get
$\mu^{\downarrow},\;\rho^{\downarrow},\;\bm{S}^{\downarrow}$. So, we get}
\begin{equation}
\rho=\left(
\begin{array}{c|c}
\rho^{\uparrow} & 0 \\
\hline
0 & \rho^{\downarrow}
\end{array}
\right)
\end{equation}
\begin{equation}
\bm{S}=\bm{S^{\uparrow}}+\bm{S^{\downarrow}}
\end{equation}
\subsection{Non-Colinear Spins Case}
\par{For the non-colinear spins case the Hamiltonian \gref{spinH} can not be anymore split in blocks($\uparrow \uparrow,\;\uparrow \downarrow,\;...$)
The dimension of the Hamiltonian now is $m=2n$, where $n$ is the number of orbitals
in the spinless case. Now the algorithms \gref{fermilevel} and \gref{matrixelements} change
to algorithm \gref{spinfermilevel} and \gref{spinmatrixelements}, respectively.}


%
\begin{algorithm}[H]
\label{spinfermilevel}
\KwData{$\{\epsilon_i\}_{i=1,m},\; a(1:m,1:m),\;T,\;k_B,\; q_{total},\;maxit$}
\KwResult{finds chemical potential $\mu$}
$a=\epsilon_1-k_BT\ln(-1+\frac{100}{q_{total}})$\;
$b=\epsilon_m+k_BT\ln(-1+\frac{100}{q_{total}})$\;
$q=-1$\;
$k=0$\;
\While{$(q \neq q_{total}) || (k < maxit)$}{
$\mu=\frac{b-a}{2}+a$\;
$q=\sum_{i=1}^{m}f_F(\epsilon_i;\mu)$\;
\eIf{$q-q_{total}>0$}{
$b=\mu$\;
}{
$a=\mu$\;
}
$k=k+1$\;
}
\If{k=maxit}{
call error("could not find $\mu$ to desired precision in $maxit$ iterations")\;
}
\caption{How to find chemical potential $\mu$ (non-colinear spins)}
\end{algorithm}
%
\begin{algorithm}
\label{spinmatrixelements} 
\caption{Density matrix elements (non-colinear spins)}
\KwData{$\{\epsilon_i\}_{i=1,m},\; a(1:m,1:m),\;T,\;k_B,\; \mu $}
\KwResult{density matrix $\rho$}
$uos=n$\;
$unos=0$\;
\For{i=1,m}{
$f(i)=f_F(\epsilon_i;\mu)$\;
\lIf(computes upper occupied state){$f(i)>q_{tolerance}$}{$uos=i$}
\tcf{computes upper occupied state with occupancy one}
\tcf{because we do not to compute $\sqrt{1}$}
\lIf{$|f(i)-1|<q_{tolerance}$}{$unos=i$}
}
\lFor{i=unos+1,uos}{$a(:,i)=\sqrt{f(i)}a(:,i)$}\;
$\rho=0$\;
compute $\rho(UpperTriangle)=a(:,1:uos)a(:,1:uos)^{\dagger*}$\;
\tcf{see algorithm \gref{uppertriangle}}
\tcf{populate lower triangle of $rho$}
\lFor{i=1,m}{$\rho(i+1:m,i)=(\rho(i,i+1:m))^*$}\;
\tcf{Compute the entropy $\bm{S}$}
$\bm{S}=0$\;
\For{i=unos+1,uos}{
$\bm{S}=\bm{S}-k_B\left(f_F(\epsilon_i)\ln(f_F(\epsilon_i)+(1-f_F(\epsilon_i))\ln(1-f_F(\epsilon_i)\right)$\;
}
\end{algorithm}

